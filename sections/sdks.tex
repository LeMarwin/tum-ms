\chapter{Changes to the toolchain}\label{chapter:sdks}

The Fabrics toolchain was also modified to enable new functionality.

As with the approach to Fabric, the approach to the toolchain was to minimize changes and to keep the original functionality intact.

\section{Fabric SDK Node}\label{sec:sdk}

Originally with fabric-sdk-node transactions get committed via \textbf{submitTransaction} function of the Contract class.

We have added a new function, called \textbf{submitUnorderedTx} which creates a transaction as per usual but uses \textbf{submitUnordered} function of the Transaction class.

\textbf{submitUnordered} is a modification of the original submit function.
It goes roughly the same route as the original one except instead of submitting the transaction to the orderer via \textbf{sendTransaction} function it sends the transaction to all peers of the channel via \textbf{sendUnorderedTransaction} function from the \textbf{Channel} class.

The \textbf{sendUnorderedTransaction} function employs the new endpoint described in \ref{sec:augm} and \textbf{Promise.allSettled} to collect all responses from the peers before resolving the promise.

\section{Caliper}\label{sec:caliper}

Caliper uses it's own functions to submit transactions so they had to be changed as well.

As with Fabric SDK Node, the changes are minimal. Caliper uses \textbf{invokeSmartContract} functions either from the blockchain.js module or from the fabric-ccp/fabric.js module, depending on the version of Fabric.

An unordered version of this function is added. \textbf{invokeSmartContractUnordered} follows the original function but uses \textbf{\_submitSingleTransactionUnordered} instead of the original one.

\textbf{\_submitSingleTransactionUnordered} again follows the original function, except it calls the \textbf{sendUnorderedTransaction} from the new fabric-sdk-node client.

In addition, some redundant checks were removed and the client now does not listen to block events from the network, since it gets confirmation of commitment directly as a result of the call.

\section{HyperLedgerLab}\label{sec:hll}

HyperLedgerLab is practically unchanged.

Two new parameters introduced in \ref{sec:bp-fin} are added to the template of a peer's configuration.

They can be provided in \textit{inventory/blockchain/group\_vars/blockchain-setup.yaml} via

\begin{lstlisting}

# Block processor's configs
fabric_peer_bp_size: "\"50\""
fabric_peer_bp_timeout: "0.2s"

\end{lstlisting}
